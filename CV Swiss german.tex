% FortySecondsCV LaTeX template
% Copyright © 2019-2020 René Wirnata <rene.wirnata@pandascience.net>
% Licensed under the 3-Clause BSD License. See LICENSE file for details.
%
% Please visit https://github.com/PandaScience/FortySecondsCV for the most
% recent version! For bugs or feature requests, please open a new issue on
% github.
%
% Contributors
% ------------
% * ifokkema
% * Bertbk
% * Hespe
%
% Attributions
% ------------
% * fortysecondscv is based on the twentysecondcv class by Carmine Spagnuolo
%   (cspagnuolo@unisa.it), released under the MIT license and available under
%   https://github.com/spagnuolocarmine/TwentySecondsCurriculumVitae-LaTex
% * further attributions are indicated immediately before corresponding code

%-------------------------------------------------------------------------------
%                             ADDITIONAL PACKAGES
%-------------------------------------------------------------------------------
\documentclass[
	a4paper,
	% showframes,
	% vline=2.2em,
	maincolor=darkestblue,
	% sidecolor=gray!50,
	sectioncolor=darkblue,
	% subsectioncolor=orange,
	% itemtextcolor=black!80,
	% sidebarwidth=0.4\paperwidth,
	% topbottommargin=0.03\paperheight,
	% leftrightmargin=20pt,
	% profilepicsize=4.5cm,
	profilepicborderwidth=2.5pt,
	% profilepicstyle=profilecircle,
	profilepiczoom=1.1,
	% profilepicxshift=0mm,
	profilepicyshift=-13mm,
	profilepicrounding=0.5cm,
]{fortysecondscv}

% improve word spacing and hyphenation
\usepackage[final]{microtype}
\usepackage{ragged2e}

% uncomment in case you don't want any hyphenation
% \usepackage[none]{hyphenat}

% take care of proper font encoding
%\ifxetexorluatex
%	\usepackage{fontspec}
%	\defaultfontfeatures{Ligatures=TeX}
%	\newfontfamily\headingfont[Path = fonts/]{NunitoSans-Italic.ttf} % local font
%\else
%	\usepackage[utf8]{inputenc}
%	\usepackage[T1]{fontenc}
%%	\usepackage[sfdefault]{noto} % use noto google font
%\fi

% enable mathematical syntax for some symbols like \varnothing
\usepackage{amssymb}

% bubble diagram configuration
\usepackage{smartdiagram}
\smartdiagramset{
	% default font size is \large, so adjust to harmonize with sidebar layout
	bubble center node font = \footnotesize,
	bubble node font = \footnotesize,
	% default: 4cm/2.5cm; make minimum diameter relative to sidebar size
	bubble center node size = 0.4\sidebartextwidth,
	bubble node size = 0.25\sidebartextwidth,
	distance center/other bubbles = 1.5em,
	% set center bubble color
	bubble center node color = maincolor!70,
	% define the list of colors usable in the diagram
	set color list = {maincolor!10, maincolor!40,
	maincolor!20, maincolor!60, maincolor!35},
	% sets the opacity at which the bubbles are shown
	bubble fill opacity = 0.8,
}

% make section titles underlined
%\titleformat{\section}{\large\bfseries}{\thesection}{1em}{\hrule}

%-------------------------------------------------------------------------------
%                            PERSONAL INFORMATION
%-------------------------------------------------------------------------------
%% mandatory information
% your name
\cvname{Laurin Tim Koller}
% job title/career
\cvjobtitle{MSc Physik, ETH Zurich}

%% optional information
% profile picture
\cvprofilepic{pics/profile.png}

% NOTE: ordering in sidebar will mimic the following order
% short address/location, use \newline if more than 1 line is required
\cvaddress{Bärengasse 3, 6317 Oberwil b. Zug}
% phone number
\cvphone{+41 079 587 16 59}
% linkedin profile
%\cvlinkedin{http://www.linkedin.com/in/laurin-koller}{www.linkedin.com/in/laurin-koller}
% github profile link
%\cvgithub{https://github.com/LaurinTim}{github.com/LaurinTim}
% email address
\cvmail{laurin.koller@protonmail.com}
% pgp key
%\cvkey{4096R/FF00FF00}{0xAABBCCDDFF00FF00}
% date of birth
\cvbirthday{1. Februar, 2000}
% any other custom entry
% nationality
\cvcustomdata{\faPassport}{Schweiz}
\cvcustomdata{\faUserFriends}{Ledig, keine Kinder}

%-------------------------------------------------------------------------------
%                              SIDEBAR 1st PAGE
%-------------------------------------------------------------------------------
% add more profile sections to sidebar on first page
\addtofrontsidebar{
    % include gosquare national flags from https://github.com/gosquared/flags;
    % naming according to ISO 3166-1 alpha-2 country codes
    \graphicspath{{pics/flags/}}

    \profilesection{Soziales Netzwerk}
        \begin{icontable}{2.0em}{1em}
            \social{\faLinkedin}
                {http://www.linkedin.com/in/laurin-koller}
                {linkedin.com/in/laurin-koller}
            \vspace{0.1cm}
            \social{\faGithub}
				{https://github.com/LaurinTim}
				{github.com/LaurinTim}
        \end{icontable}
    
    \profilesection{Sprachen}
        \pointskill{\hspace{0.8cm} Deutsch}{}{5}
        \pointskill{\hspace{0.8cm} Englisch}{}{5}
        \pointskill{\hspace{1.4cm} Französisch}{}{3}
    
    \profilesection{Hard Skills}
            %\barskill{\faDesktop}{Computer skills}{100}
        \barskill{\faPython}{Python}{100}
                \skill[1.0em]{\faAngleRight}{NumPy}
                \skill[1.0em]{\faAngleRight}{Pandas}
                \skill[1.0em]{\faAngleRight}{PyTorch}
                %\skill[1.0em]{\faAngleRight}{PySpark}
        \barskill{\faDatabase}{SQL}{80}
        \barskill{\faKeyboard}{MS office}{80}
        \barskill{\faDesktop}{Künstliche Intelligenz}{70}
            \skill[1.0em]{\faAngleRight}{Maschinelles lernen}
            \skill[1.0em]{\faAngleRight}{Deep learning}
            \skill[1.0em]{\faAngleRight}{Natural language processing}
        \barskill{\faCode}{C++}{50}
}

%-------------------------------------------------------------------------------
%                              SIDEBAR 2nd PAGE
%-------------------------------------------------------------------------------
\addtobacksidebar{
    \profilesection{Soft Skills}
        \skill[0.0em]{\faUsers}{\textit{Zusammenarbeit}}
            \skill[0.8em]{\faAngleRight}{Arbeiten in agilen Teams}
            \skill[0.8em]{\faAngleRight}{Funktionsübergreifende Zusammenarbeit}
        \skill[0.0em]{\faUserCog}{\textit{Problemlösung}}
            \skill[0.8em]{\faAngleRight}{Entwerfen von ML-Modellen}
            \skill[0.8em]{\faAngleRight}{Optimierung von Daten-Workflowss}
        \skill[0.0em]{\faComments}{\textit{Kommunikation}}
            \skill[0.8em]{\faAngleRight}{Verfassen technischer Berichte}
            \skill[0.8em]{\faAngleRight}{Präsentation neuer Erkenntnisse vor interdisziplinären Teams}
        \skill[0.0em]{\faFlask}{\textit{Forschung}}


    \profilesection{Hobbies}
        \skill[0em]{\faPlaneDeparture}{Reisen}
        \skill[0em]{\faCookieBite}{Kochen und Backen}
        \skill[0em]{\faHiking}{Wandern}
        \skill[0em]{\faGamepad}{Videospiele}
        \skill[0em]{\faComment}{Japanisch lernen}
        \skill[0em]{\faMusic}{Klavier spielen}
}

%-------------------------------------------------------------------------------
%                         TABLE ENTRIES RIGHT COLUMN
%-------------------------------------------------------------------------------
\begin{document}

\makefrontsidebar

%\vspace*{-10pt}

\cvsection{Über mich} \vspace{-0.1cm}
\begin{cvtable}
    \cvitem{}{}{}{
        \textpoint{\faAngleRight}{Leidenschaft für Problemlösungen, Datenanalyse und maschinelles Lernen und deren Anwendung auf reale Herausforderungen.}
        \textpoint{\faAngleRight}{Analytisch, strukturiert und gut im Umgang mit Zahlen und Daten.}
        \textpoint{\faAngleRight}{Über fünf Jahre Erfahrung mit Programmierung - hauptsächlich Python, aber auch SQL und C++.}
        \textpoint{\faAngleRight}{Über fünf Jahre Erfahrung mit programmieren - hauptsächlich Python aber auch SQL und C++.}
        \textpoint{\faAngleRight}{Vier Jahre Erfahrung in der Arbeit mit Datensätzen in Python.}
        \textpoint{\faAngleRight}{Praktische Erfahrung mit maschinellem Lernen und Deep Learning sowie Erfahrung in der Implementierung von Modellen.}
        \textpoint{\faAngleRight}{Kenntnisse der Grundlagen der Verarbeitung natürlicher Sprache und generativer KI.}
        \textpoint{\faAngleRight}{Effektiver Teamplayer mit starken Kommunikationsfähigkeiten und Anpassungsfähigkeit in kollaborativen Umgebungen.}
    }
\end{cvtable}

\cvsection{Forschungserfahrung} \vspace{-0.6cm}
\begin{cvtable} [3]
    \cvitem{March 2024 -- October 2024}{Masterarbeit Forscher beim GBAR Experiment (CERN),}{Genf}{
        \textpoint{\faAngleRight}{Entwicklung und Implementierung von Datenintegrationspipelines unter Verwendung von pandas und der Apache Spark-Architektur des CERN.} 
        \textpoint{\faAngleRight}{Trainierte und optimierte maschinelle Lernmodelle zur Klassifizierung von Einzelteilchen- und Photoneneinschlägen auf der Grundlage von Energiedepositionsmustern.} 
        \textpoint{\faAngleRight}{Analysierte die Produktionsraten von Antiwasserstoff am GBAR-Experiment und erzielte eine vierfache Verbesserung gegenüber früheren Studien.}
        \textpoint{\faAngleRight}{Verwendete statistische Modellierung, Signalverarbeitung und Big-Data-Techniken an, um Forschungsabläufe zu optimieren.}
        \textpoint{\faAngleRight}{Entdeckung des ersten Nachweises von Lyman-Alpha Photonen unter Verwendung des GBAR-Setups mit einem ionisierten Wasserstoffstrahl, was zu präzisen Antimaterie-Messungen beiträgt.}
    }
\end{cvtable}

\cvsection{Ausbildung} \vspace{-0.6cm}
\begin{cvtable}[3]
    \vspace{-0.5cm}
    \cvitem{2022 -- 2024}{MSc Physik,}{ETH Zurich}{
        \textpoint{\faAngleRight}{Wahlfächer in Teilchenphysik, Astrophysik, Quantenfeldtheorie und Makroökonomie.}
        \textpoint{\faAngleRight}{Proseminar in theoretischer Physik mit einem Projekt in shape dynamics.}
    }
    \vspace{-0.5cm}
    \cvitem{2019 -- 2023}{BSc Physik,}{ETH Zurich}{
        \textpoint{\faAngleRight}{Wahlfächer in Teilchenphysik, Quantenmechanik, klassischer Mechanik und Geschichte.}
        \textpoint{\faAngleRight}{Führte u. a. Experimente zu Fraktalclustern, Kohlenstoffdatieren und Simulationen von Virenübertragung durch.}
    }
    \vspace{-0.5cm}
    \cvitem{Januar 2019 -- Mai 2019}{Training als Nuklearer Labortechniker, Schweizer Militär,}{Spiez}{
        \textpoint{\faAngleRight}{Durchführung von Alpha- und Betastrahlungsmessungen an Lebensmittelproben, um deren Unbedenklichkeit für den Verzehr zu prüfen.}
        \textpoint{\faAngleRight}{Messung von Gammastrahlungsproben zur Bestimmung ihrer Zusammensetzung an radioaktiven Isotopen.}
    }
    \vspace{-0.5cm}
    \cvitem{2012 -- 2018}{Matura,}{Kantonsschule Zug}{
        \textpoint{\faAngleRight}{Matura mit den Schwerpunkten Mathematik, Physik und Informatik.}
        \textpoint{\faAngleRight}{Abschlussprojekt zur Messung der atomaren Struktur verschiedener Sternen anhand ihres Spektrums im sichtbarem Licht.}
        %\textpoint{\faAngleRight}{}
    }
\end{cvtable}

%-------------------------------------------------------------------------------
%                              TABLE ENTRIES 2nd PAGE
%-------------------------------------------------------------------------------

\newpage
\makebacksidebar

\vspace*{-10pt}

\cvsection{Ausbildungsbegleitende Arbeit} \vspace{-0.6cm}
\begin{cvtable}[3]
    \vspace{-0.5cm}
    \cvitem{Juli 2018 -- August 2018}{Studenten Job bei Coop,}{Baar}{
        \textpoint{\faAngleRight}{Arbeitete als Kassierer bei Coop.}
    }
    \vspace{-0.5cm}
    \cvitem{Juli 2016 -- August 2016}{Studenten Job bei V-Zug,}{Zug}{
        \textpoint{\faAngleRight}{Durchführung von Qualitätskontrollen an Maschinenteilen.}
        \textpoint{\faAngleRight}{Entgegennahme und Weiterleitung neuer Lieferungen an das Lager.}
    }
    \cvitem{Juli 2015 -- August 2015}{Hausmeister in der Grundschule Baar,}{Baar}{}
\end{cvtable}

\end{document}

\vspace{-0.5cm}
    \cvitem{2012 -- 2018}{Matura,}{Kantonsschule Zug}{
        \textpoint{\faAngleRight}{Matura mit den Schwerpunkten Mathematik, Physik und Informatik.}
        \textpoint{\faAngleRight}{Abschlussprojekt zur Messung der atomaren Struktur verschiedener Sternen anhand ihres Spektrums im sichtbarem Licht.}
        %\textpoint{\faAngleRight}{}
    }