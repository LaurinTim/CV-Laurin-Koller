% FortySecondsCV LaTeX template
% Copyright © 2019-2020 René Wirnata <rene.wirnata@pandascience.net>
% Licensed under the 3-Clause BSD License. See LICENSE file for details.
%
% Please visit https://github.com/PandaScience/FortySecondsCV for the most
% recent version! For bugs or feature requests, please open a new issue on
% github.
%
% Contributors
% ------------
% * ifokkema
% * Bertbk
% * Hespe
%
% Attributions
% ------------
% * fortysecondscv is based on the twentysecondcv class by Carmine Spagnuolo
%   (cspagnuolo@unisa.it), released under the MIT license and available under
%   https://github.com/spagnuolocarmine/TwentySecondsCurriculumVitae-LaTex
% * further attributions are indicated immediately before corresponding code

%-------------------------------------------------------------------------------
%                             ADDITIONAL PACKAGES
%-------------------------------------------------------------------------------
\documentclass[
	a4paper,
	% showframes,
	% vline=2.2em,
	maincolor=darkestblue,
	% sidecolor=gray!50,
	sectioncolor=darkblue,
	% subsectioncolor=orange,
	% itemtextcolor=black!80,
	% sidebarwidth=0.4\paperwidth,
	% topbottommargin=0.03\paperheight,
	% leftrightmargin=20pt,
	% profilepicsize=4.5cm,
	profilepicborderwidth=2.5pt,
	% profilepicstyle=profilecircle,
	profilepiczoom=1.1,
	% profilepicxshift=0mm,
	profilepicyshift=-13mm,
	profilepicrounding=0.5cm,
]{fortysecondscv}

% improve word spacing and hyphenation
\usepackage{microtype}
\usepackage{ragged2e}

% uncomment in case you don't want any hyphenation
% \usepackage[none]{hyphenat}

% take care of proper font encoding
\ifxetexorluatex
	\usepackage{fontspec}
	\defaultfontfeatures{Ligatures=TeX}
%	\newfontfamily\headingfont[Path = fonts/]{segoeuib.ttf} % local font
\else
	\usepackage[utf8]{inputenc}
	\usepackage[T1]{fontenc}
%	\usepackage[sfdefault]{noto} % use noto google font
\fi

% enable mathematical syntax for some symbols like \varnothing
\usepackage{amssymb}

% bubble diagram configuration
\usepackage{smartdiagram}
\smartdiagramset{
	% default font size is \large, so adjust to harmonize with sidebar layout
	bubble center node font = \footnotesize,
	bubble node font = \footnotesize,
	% default: 4cm/2.5cm; make minimum diameter relative to sidebar size
	bubble center node size = 0.4\sidebartextwidth,
	bubble node size = 0.25\sidebartextwidth,
	distance center/other bubbles = 1.5em,
	% set center bubble color
	bubble center node color = maincolor!70,
	% define the list of colors usable in the diagram
	set color list = {maincolor!10, maincolor!40,
	maincolor!20, maincolor!60, maincolor!35},
	% sets the opacity at which the bubbles are shown
	bubble fill opacity = 0.8,
}

% make section titles underlined
%\titleformat{\section}{\large\bfseries}{\thesection}{1em}{\hrule}

%-------------------------------------------------------------------------------
%                            PERSONAL INFORMATION
%-------------------------------------------------------------------------------
%% mandatory information
% your name
\cvname{Laurin Tim Koller}
% job title/career
\cvjobtitle{MSc Physics, ETH Zurich}

%% optional information
% profile picture
%\cvprofilepic{pics/profile.png}

% NOTE: ordering in sidebar will mimic the following order
% short address/location, use \newline if more than 1 line is required
\cvaddress{Bärengasse 3, 6317 Oberwil b. Zug}
% phone number
\cvphone{+41 079 587 16 59}
% email address
\cvmail{laurin.koller@protonmail.com}
% linkedin profile
\cvlinkedin{http://www.linkedin.com/in/laurin-koller}{www.linkedin.com/in/laurin-koller}
% github profile link
%\cvgithub{https://github.com/LaurinTim}{github.com/LaurinTim}
% pgp key
%\cvkey{4096R/FF00FF00}{0xAABBCCDDFF00FF00}
% date of birth
%\cvbirthday{February 1, 2000}
% any other custom entry
% nationality
%\cvcustomdata{\faFlag}{Swiss}

%-------------------------------------------------------------------------------
%                              SIDEBAR 1st PAGE
%-------------------------------------------------------------------------------
% add more profile sections to sidebar on first page
\addtofrontsidebar{
    % include gosquare national flags from https://github.com/gosquared/flags;
    % naming according to ISO 3166-1 alpha-2 country codes
    \graphicspath{{pics/flags/}}

    %\profilesection{Social Network}
    %    \begin{icontable}{2.0em}{1em}
    %        \social{\faLinkedin}
    %            {http://www.linkedin.com/in/laurin-koller}
    %            {linkedin.com/in/laurin-koller}
    %        \vspace{0.1cm}
    %        \social{\faGithub}
	%			{https://github.com/LaurinTim}
	%			{github.com/LaurinTim}
    %    \end{icontable}
    
    \profilesection{Languages}
        \pointskill{\hspace{0.8cm} German}{}{5}
        \pointskill{\hspace{0.8cm} English}{}{5}
        \pointskill{\hspace{0.8cm} French}{}{3}
    
    \profilesection{Hard Skills}
            %\barskill{\faDesktop}{Computer skills}{100}
        \barskill{\faPython}{\large{Python}}{100}
                \skill[1.0em]{\faAngleRight}{NumPy}
                \skill[1.0em]{\faAngleRight}{Pandas}
                \skill[1.0em]{\faAngleRight}{PyTorch}
                %\skill[1.0em]{\faAngleRight}{PySpark}
        \barskill{\faDatabase}{\large{SQL}}{80}
        %\barskill{\faKeyboard}{MS Office}{80}
        \barskill{\faDesktop}{\large{Artificial Intelligence}}{70}
            \skill[1.0em]{\faAngleRight}{Machine learning}
            \skill[1.0em]{\faAngleRight}{Natural language processing}
            \skill[1.0em]{\faAngleRight}{Deep learning}
        \barskill{\faCode}{\large{C++}}{50}

    \profilesection{Soft Skills}
        \skill{\faUserCog}{Problem solving}
        \skill{\faFlask}{Research}
        \skill{\faExclamation}{Critical thinking}
        \skill{\faComments}{Communication}
        \skill{\faUsers}{Collaboration}
}

%-------------------------------------------------------------------------------
%                         TABLE ENTRIES RIGHT COLUMN
%-------------------------------------------------------------------------------
\begin{document}

\makefrontsidebar

%\vspace*{-10pt}

\cvsection{About me} \vspace{-0.2cm}
\begin{cvtable}
    \cvitem{}{}{}{
        \textpoint{\faAngleRight}{Passionate about problem solving, data analysis and machine learning and how to apply them to real-world challenges.}
        \textpoint{\faAngleRight}{Analytical, structured and good with numbers and data.}
        \textpoint{\faAngleRight}{Over five years of experience with programming - mainly Python but also SQL and C++.}
        \textpoint{\faAngleRight}{Four years of experience working with datasets in Python.}
        \textpoint{\faAngleRight}{Hands-on expertise in machine learning and deep learning, with direct experience implementing models.}
        \textpoint{\faAngleRight}{Knowledge of the basics of natural language processing and generative AI.}
        \textpoint{\faAngleRight}{Effective team player with strong communication skills and adaptability in collaborative environments.}
    }
\end{cvtable}

\vspace{0.3cm}

\cvsection{Research Experience} \vspace{-0.8cm}
\begin{cvtable}[3]
    \cvitem{March 2024 -- October 2024}{Master Thesis Researcher at the GBAR Experiment (CERN),}{Geneva}{
        \textpoint{\faAngleRight}{Developed \& implemented data integration pipelines using pandas and CERN’s Apache Spark architecture.} 
        \textpoint{\faAngleRight}{Trained and optimized machine learning models to classify single-particle and photon impacts based on energy deposition patterns.} 
        \textpoint{\faAngleRight}{Analyzed antihydrogen production rates at the GBAR experiment, achieving a 4x improvement over previous studies.} 
        \textpoint{\faAngleRight}{Applied statistical modeling, signal processing and big data techniques to optimize research workflows.}
        \textpoint{\faAngleRight}{Discovered first evidence of Lyman alpha photons using GBAR’s setup with an ionized hydrogen beam, contributing to fundamental antimatter research.}
    }
\end{cvtable}

\vspace{-0.2cm}

\cvsection{Education} \vspace{-0.8cm}
\begin{cvtable}[3]
    \vspace{-0.4cm}
    \cvitem{2022 -- 2024}{MSc Physics,}{ETH Zurich}{
        \textpoint{\faAngleRight}{Elective courses in Particle physics, Astrophysics, Quantum field theory and Macroeconomics.}
        \textpoint{\faAngleRight}{Project in theoretical physics about Shape dynamics.}
    }
    \vspace{-0.4cm}
    \cvitem{2019 -- 2023}{BSc Physics,}{ETH Zurich}{
        \textpoint{\faAngleRight}{Elective courses in Particle physics, Quantum mechanics, Classical mechanics and History.}
        \textpoint{\faAngleRight}{Conducted, among others, experiments in fractal clusters, carbon dating, and virus transmission simulations.}
    }
    \vspace{-0.4cm}
    \cvitem{January 2019 -- May 2019}{Nuclear Laboratory Technician Training, Swiss Military,}{Spiez}{
        \textpoint{\faAngleRight}{Conducted alpha and beta radiation measurements on food samples for safety assessment.}
        \textpoint{\faAngleRight}{Analyzed gamma radiation to determine isotopic composition.}
    }
    \vspace{-0.4cm}
    \cvitem{2012 -- 2018}{Matura,}{Kantonsschule Zug}{
        \textpoint{\faAngleRight}{Specialization: Mathematics, Physics and Computer Science.}
        \textpoint{\faAngleRight}{Final project measuring the atomic structure of different stars using their spectrum of visible light.}
        %\textpoint{\faAngleRight}{}
    }
\end{cvtable}

\end{document}

\cvsection{Other experience} \vspace{-0.8cm}
\begin{cvtable}[3]
    \vspace{-0.5cm}
    \cvitem{July 2018 -- August 2018}{Retail Cashier,}{Coop: Customer service and checkout operations.}{
        %\textpoint{\faAngleRight}{Customer service and checkout operations.}
    }
    \vspace{-0.5cm}
    \cvitem{July 2016 -- August 2016}{Intern, V-Zug: Conducted quality control and managed inventory.}{}{
        %\textpoint{\faAngleRight}{Performed quality control checks of machine parts.}
        %\textpoint{\faAngleRight}{Received and sent on new deliveries to the warehouse.}
    }
\end{cvtable}