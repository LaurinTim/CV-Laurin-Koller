% FortySecondsCV LaTeX template
% Copyright © 2019-2020 René Wirnata <rene.wirnata@pandascience.net>
% Licensed under the 3-Clause BSD License. See LICENSE file for details.
%
% Please visit https://github.com/PandaScience/FortySecondsCV for the most
% recent version! For bugs or feature requests, please open a new issue on
% github.
%
% Contributors
% ------------
% * ifokkema
% * Bertbk
% * Hespe
%
% Attributions
% ------------
% * fortysecondscv is based on the twentysecondcv class by Carmine Spagnuolo
%   (cspagnuolo@unisa.it), released under the MIT license and available under
%   https://github.com/spagnuolocarmine/TwentySecondsCurriculumVitae-LaTex
% * further attributions are indicated immediately before corresponding code

%-------------------------------------------------------------------------------
%                             ADDITIONAL PACKAGES
%-------------------------------------------------------------------------------
\documentclass[
	a4paper,
	% showframes,
	% vline=2.2em,
	maincolor=darkestblue,
	% sidecolor=gray!50,
	sectioncolor=darkblue,
	% subsectioncolor=orange,
	% itemtextcolor=black!80,
	% sidebarwidth=0.4\paperwidth,
	% topbottommargin=0.03\paperheight,
	% leftrightmargin=20pt,
	% profilepicsize=4.5cm,
	profilepicborderwidth=2.5pt,
	% profilepicstyle=profilecircle,
	profilepiczoom=1.1,
	% profilepicxshift=0mm,
	profilepicyshift=-13mm,
	profilepicrounding=0.5cm,
]{fortysecondscv}

% improve word spacing and hyphenation
\usepackage[final]{microtype}
\usepackage{ragged2e}

% uncomment in case you don't want any hyphenation
% \usepackage[none]{hyphenat}

% take care of proper font encoding
%\ifxetexorluatex
%	\usepackage{fontspec}
%	\defaultfontfeatures{Ligatures=TeX}
%	\newfontfamily\headingfont[Path = fonts/]{NunitoSans-Italic.ttf} % local font
%\else
%	\usepackage[utf8]{inputenc}
%	\usepackage[T1]{fontenc}
%%	\usepackage[sfdefault]{noto} % use noto google font
%\fi

% enable mathematical syntax for some symbols like \varnothing
\usepackage{amssymb}

% bubble diagram configuration
\usepackage{smartdiagram}
\smartdiagramset{
	% default font size is \large, so adjust to harmonize with sidebar layout
	bubble center node font = \footnotesize,
	bubble node font = \footnotesize,
	% default: 4cm/2.5cm; make minimum diameter relative to sidebar size
	bubble center node size = 0.4\sidebartextwidth,
	bubble node size = 0.25\sidebartextwidth,
	distance center/other bubbles = 1.5em,
	% set center bubble color
	bubble center node color = maincolor!70,
	% define the list of colors usable in the diagram
	set color list = {maincolor!10, maincolor!40,
	maincolor!20, maincolor!60, maincolor!35},
	% sets the opacity at which the bubbles are shown
	bubble fill opacity = 0.8,
}

% make section titles underlined
%\titleformat{\section}{\large\bfseries}{\thesection}{1em}{\hrule}

%-------------------------------------------------------------------------------
%                            PERSONAL INFORMATION
%-------------------------------------------------------------------------------
%% mandatory information
% your name
\cvname{Laurin Tim Koller}
% job title/career
\cvjobtitle{MSc Physics, ETH Zurich}

%% optional information
% profile picture
\cvprofilepic{pics/profile.png}

% NOTE: ordering in sidebar will mimic the following order
% short address/location, use \newline if more than 1 line is required
\cvaddress{Bärengasse 3, 6317 Oberwil b. Zug}
% phone number
\cvphone{+41 079 587 16 59}
% linkedin profile
%\cvlinkedin{http://www.linkedin.com/in/laurin-koller}{www.linkedin.com/in/laurin-koller}
% github profile link
%\cvgithub{https://github.com/LaurinTim}{github.com/LaurinTim}
% email address
\cvmail{laurin.koller@protonmail.com}
% pgp key
%\cvkey{4096R/FF00FF00}{0xAABBCCDDFF00FF00}
% date of birth
\cvbirthday{February 1, 2000}
% any other custom entry
% nationality
\cvcustomdata{\faPassport}{Swiss}

%-------------------------------------------------------------------------------
%                              SIDEBAR 1st PAGE
%-------------------------------------------------------------------------------
% add more profile sections to sidebar on first page
\addtofrontsidebar{
    % include gosquare national flags from https://github.com/gosquared/flags;
    % naming according to ISO 3166-1 alpha-2 country codes
    \graphicspath{{pics/flags/}}

    \profilesection{Social Network}
        \begin{icontable}{2.0em}{1em}
            \social{\faLinkedin}
                {http://www.linkedin.com/in/laurin-koller}
                {linkedin.com/in/laurin-koller}
            \vspace{0.1cm}
            %\social{\faGithub}
		%		{https://github.com/LaurinTim}
		%		{github.com/LaurinTim}
        \end{icontable}
    
    \profilesection{Languages}
        \pointskill{\hspace{0.8cm} German}{}{5}
        \pointskill{\hspace{0.8cm} English}{}{5}
        \pointskill{\hspace{0.8cm} French}{}{3}
    
    \profilesection{Hard Skills}
            %\barskill{\faDesktop}{Computer skills}{100}
        \barskill{\faPython}{Python}{100}
                \skill[1.0em]{\faAngleRight}{NumPy}
                \skill[1.0em]{\faAngleRight}{Pandas}
                \skill[1.0em]{\faAngleRight}{PyTorch}
        \barskill{\faDatabase}{SQL}{80}
        \barskill{\faKeyboard}{MS office}{80}
        \barskill{\faDesktop}{Artificial Intelligence}{70}
            \skill[1.0em]{\faAngleRight}{Machine learning}
            \skill[1.0em]{\faAngleRight}{Natural language processing}
        \barskill{\faCode}{C++}{50}
}

%-------------------------------------------------------------------------------
%                              SIDEBAR 2nd PAGE
%-------------------------------------------------------------------------------
\addtobacksidebar{
    \profilesection{Soft Skills}
        \barskill{\faUserCog}{Problem solving}{100}
        \barskill{\faComments}{Communication}{80}
        \barskill{\faUsers}{Collaboration}{80}

    \profilesection{Hobbies}
        \skill[0em]{\faPlaneDeparture}{Traveling}
        \skill[0em]{\faCookieBite}{Cooking and Baking}
        \skill[0em]{\faHiking}{Hiking}
        \skill[0em]{\faGamepad}{Video Games}
        \skill[0em]{\faComment}{Learning Japanese}
        \skill[0em]{\faMusic}{Playing the Piano}
}

%-------------------------------------------------------------------------------
%                         TABLE ENTRIES RIGHT COLUMN
%-------------------------------------------------------------------------------
\begin{document}

\makefrontsidebar

\vspace*{-10pt}

\cvsection{About me} \vspace{0.2cm}
\begin{cvtable}
    \cvitem{}{}{}{
        \textpoint{\faAngleRight}{I always strive to learn new things and to apply them to real-life situations.}
        \textpoint{\faAngleRight}{Analytical, structured and good with numbers and data.}
        \textpoint{\faAngleRight}{Several years of experience with programming - mainly Python but also SQL and C++.}
        \textpoint{\faAngleRight}{Four years of experience working with datasets in Python.}
        \textpoint{\faAngleRight}{Solid knowledge of the basics of machine learning and natural language processing and a year of experience working with machine learning models.}
        \textpoint{\faAngleRight}{I work well in teams thanks to my flexibility and communication.}
    }
\end{cvtable}

\vspace{0.5cm}

\cvsection{Education} \vspace{-0.3cm}
\begin{cvtable}[3]
    %\vspace{-0.2cm}
    \cvitem{March 2024 -- October 2024}{Master Thesis Researcher at the GBAR Experiment (CERN),}{Geneva}{
        \textpoint{\faAngleRight}{Independently developed and implemented routines for the data integration of multiple subsystems of the GBAR experiment.} 
        \textpoint{\faAngleRight}{Successfully trained machine learning models to extract single particle and photon impacts from their deposited charge on a phosphor screen.} 
        \textpoint{\faAngleRight}{Determined production rate of antihydrogen at the GBAR experiment and successfully showed that this quantity is at least four times higher than previous analyses calculated.} 
        \textpoint{\faAngleRight}{Analyzed and recorded experimental data for the Lamb shift in hydrogen and found first evidence of Lyman alpha photons using the setup of the GBAR experiment and an ionized hydrogen beam.}
    }
    \vspace{-0.1cm}
    \cvitem{2022 -- 2024}{MSc Physics,}{ETH Zurich}{
        \textpoint{\faAngleRight}{Elective courses in Particle physics, Astrophysics, Quantum field theory and Macroeconomics.}
        \textpoint{\faAngleRight}{Proseminar in theoretical physics with a project in Shape dynamics.}
    }
    \vspace{-0.1cm}
    \cvitem{2019 -- 2023}{BSc Physics,}{ETH Zurich}{
        \textpoint{\faAngleRight}{Elective courses in Particle physics, Quantum mechanics, Classical mechanics and History.}
        \textpoint{\faAngleRight}{Conducted, among others, experiments about Fractal clusters, Carbon dataing and Virus transmission simulations.}
    }
    \vspace{-0.1cm}
    \cvitem{January 2019 -- May 2019}{Nuclear Laboratory Technician in the Swiss Military,}{Spiez}{
        \textpoint{\faAngleRight}{Conducted alpha and beta radiation measurements of food samples to asses their safety for consumption.}
        \textpoint{\faAngleRight}{Measured gamma radiation samples to determine their composition of radioactive isotopes.}
    }
    \vspace{-0.1cm}
    \cvitem{2012 -- 2018}{Matura,}{Kantonsschule Zug}{
        \textpoint{\faAngleRight}{Matura with a focus on mathematics, physics and computer science}
        \textpoint{\faAngleRight}{Final project measuring the atomic structure of different stare using their spectrum of visible light.}
        %\textpoint{\faAngleRight}{}
    }
\end{cvtable}

%-------------------------------------------------------------------------------
%                              TABLE ENTRIES 2nd PAGE
%-------------------------------------------------------------------------------

\newpage
\makebacksidebar

\vspace*{-10pt}

\cvsection{Work history during studies} \vspace{-0.3cm}
\begin{cvtable}[3]
    \cvitem{July 2018 -- August 2018}{Student job at Coop,}{Baar}{
        \textpoint{\faAngleRight}{Worked as a cashier for Coop.}
    }
    \vspace{-0.1cm}
    \cvitem{July 2016 -- August 2016}{Student Job at V-Zug,}{Zug}{
        \textpoint{\faAngleRight}{Performed quality control checks of machine parts.}
        \textpoint{\faAngleRight}{Received and sent on new deliveries to the warehouse.}
    }
    \cvitem{July 2015 -- August 2015}{Janitor at Primary School Baar,}{Baar}{}
\end{cvtable}

\end{document}
